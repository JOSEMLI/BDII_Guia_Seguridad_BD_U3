\section{parte 01 – Introduccion} 

\begin{enumerate}[1.]
	\item INTRODUCCION\\
	\\En el presente informe Se explicara cómo es que se debe realizar un respaldo de la información, en este caso el respaldo de una base de datos en Oracle 11g Enterprise Edition para el uso del asistente grafico para copias de seguridad (Enterprise Manager).
Además se utilizara SQLDEVElOPER.exe para para conectar un usuario, también sirve para migración de bases de datos de MySQL a Oracle.
Se explicara qué tipos de backups se pueden realizar en Oracle, algunas recomendaciones de cuando realizar las copias de seguridad además de copias de seguridad en modo consola y de manera grafica.
\\
\\Una copia de los datos que se puede utilizar para restaurar y recuperar los datos se denomina copia de seguridad. Las copias de seguridad le permiten restaurar los datos después de un error. Con las copias de seguridad correctas, puede recuperarse de multitud de errores como:\\
\\
-  Errores de medios.\\
-  Errores de usuario, por ejemplo, quitar una tabla por error.\\
- Errores de hardware, por ejemplo, una unidad de disco dañada o la pérdida Permanente de un servidor.\\
- Desastres naturales.\\

Además, las copias de seguridad de una base de datos son útiles para fines administrativos habituales, como copiar una base de datos de un servidor a otro, configurar la creación de reflejo de la base de datos y el archivo, etc.
\\

\vspace*{0.8in}
\begin{Large}
\textbf{Guía de Implementación de estrategia de Copias se Seguridad y Recuperación de base de Datos} \\
\end{Large}



	
	

\end{enumerate} 
