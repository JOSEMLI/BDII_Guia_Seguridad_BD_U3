\section{parte 03 –  Marco Teórico} 

\begin{enumerate}[3.1.]
	\item COPIAS DE SEGURIDAD Y RESTAURACIÓN DE BASES DE DATOS
	

Una copia de los datos que se puede utilizar para restaurar y recuperar los datos se denomina copia de seguridad. Las copias de seguridad le permiten restaurar los datos después de un error. Con las copias de seguridad correctas, puede recuperarse de multitud de errores, por ejemplo:\\

- Errores de medios.\\
- Errores de usuario, por ejemplo, quitar una tabla por error.\\
- Errores de hardware, por ejemplo, una unidad de disco dañada o la pérdida\\
- permanente de un servidor.\\
- Desastres naturales.\\

Además, las copias de seguridad de una base de datos son útiles para fines administrativos habituales, como copiar una base de datos de un servidor a otro, configurar la creación de reflejo de la base de datos y el archivo, etc.\\
	

\item COMO IMPEDIR LA PERDIDA DE DATOS\\
\\	
Impedir la pérdida de datos es uno de los problemas más importantes que afrontan los administradores de sistemas.\\

a) Disponer de una estrategia de copia de seguridad

Debe tener una estrategia de copia de seguridad para aminorar la pérdida de datos y recuperar los datos perdidos. Los datos se pueden perder como consecuencia de errores de hardware o de software, o bien por:\\

- El uso accidental o malintencionado de una instrucción DELETE.\\
- El uso accidental o malintencionado de una instrucción UPDATE; por ejemplo, no utilizar la cláusula WHERE con una instrucción\\ UPDATE (se actualizan todas las filas en lugar de una fila concreta de la tabla).\\
- Virus destructivos.\\
- Desastres naturales, como incendios, inundaciones y terremotos.\\
- Robo.\\

Si utiliza una estrategia de copia de seguridad adecuada, puede restaurar los datos con un costo mínimo sobre la producción y reducir la posibilidad de que los datos se pierdan definitivamente. Piense en la estrategia de copia de seguridad como un seguro. Su estrategia de copia de seguridad debe devolver el sistema al punto en el que se encontraba antes del problema. Al igual que con una 


\end{enumerate} 
